\makeatletter
%____________________________vertical bar________________________________
\newtoks \coords
%%%%%%%%%%%%%%%%%%%%%%%%%%%%%%%%%%%  dimensions  %%%%%%%%%%%%%%%%%%%%%%%%%%%%%%%%%%%%%%%%%%%
\define@key{VerticalBar}{Width}{\def\VerticalBarwidth{#1}} % Width of Chart
\define@key{VerticalBar}{Height}{\def\VerticalBarheight{#1}}% Height of Chart
\define@key{VerticalBar}{Barwidth}{\def\VerticalBarbarwidth{#1}}% BarWidth of Chart
\define@key{VerticalBar}{Textwidth}{\def\VerticalBartextwidth{#1}}% Text width from axisline
\define@key{VerticalBar}{Scale}{\def\VerticalBarscale{#1}}% Size of Chart 
\define@key{VerticalBar}{YMax}{\def\VerticalBarYMax{#1}} % Y Axis Max Value
\define@key{VerticalBar}{YMin}{\def\VerticalBarYMin{#1}} % Y Axis Min Value

%%%%%%%%%%%%%%%%%%%%%%%%%%%%%%%%%%%%%% labels  %%%%%%%%%%%%%%%%%%%%%%%%%%%%%%%%%%%%%%%%%%%%%
% \define@key{VerticalBar}{Xticklabels}{\def\VerticalBarXticklabels{#1}} % X axis Tick Labels 
\define@key{VerticalBar}{Ylabels}{\def\VerticalBarYlabels{#1}} % Y axis labels
\define@key{VerticalBar}{Xlabel}{\def\VerticalBarXlabel{#1}} % X axis label
\define@key{VerticalBar}{Xlabelrotate}{\def\VerticalBarXlabelrotate{#1}} % X tick label rotate
%%%%%%%%%%%%%%%%%%%%%%%%%%%%%%%%%%%%%% Axis Values  %%%%%%%%%%%%%%%%%%%%%%%%%%%%%%%%%%%%%%%%%%%%%
\define@key{VerticalBar}{Xtick}{\def\VerticalBarXtick{#1}}% X axis tick values
\define@key{VerticalBar}{Ytick}{\def\VerticalBarYtick{#1}}% Y axis tick values
\define@key{VerticalBar}{Barcoords}{\def\VerticalBarcoords{#1}}% Bar cordinates

%%%%%%%%%%%%%%%%%%%%%%%%%%%%%%%%%%%%%%%%%%%%%%%%%%%%%%%%%%%%%%%%%%%%%%%%%%%%%%%%%%%%%%%%%%%%%%%%
\define@key{VerticalBar}{NodeFontsize}{\def\VerticalBarNodeFontSize{#1}}
\define@key{VerticalBar}{Fontsize}{\def\VerticalBarFontSize{#1}}
\define@key{VerticalBar}{Title}{\def\VerticalBartitle{#1}} % Title of Chart
\define@key{VerticalBar}{Suffix}{\def\suffix{#1}} %add Suffix to Average line node value
\define@key{VerticalBar}{Prefix}{\def\prefix{#1}} %add Prefix to Average line node value
\define@key{VerticalBar}{Align}{\def\VerticalBarAlign{#1}} %align X tick label
\define@key{VerticalBar}{Anchor}{\def\VerticalBarAnchor{#1}} %anchor for x tick label
\define@key{VerticalBar}{Yshift}{\def\VerticalBarYshift{#1}} % vertical positioning of x tick label
\define@key{VerticalBar}{Xshift}{\def\VerticalBarXshift{#1}} % horizontal positioning of x tick label
\define@key{VerticalBar}{LegendTopBottom}{\def\VerticalBarLegendYposition{#1}} % vertical positioning of legend
\define@key{VerticalBar}{LegendLeftRight}{\def\VerticalBarLegendXposition{#1}} % horizontal positioning of legend
\define@key{VerticalBar}{ColorChoice}{\def\choice{#1}} % to choose digital or print color mode
\define@key{VerticalBar}{Xenlarge}{\def\VerticalBarXlimitEnLarge{#1}}
\define@key{VerticalBar}{TitleUpDown}{\def\verticalbartitleupdown{#1}}
\define@key{VerticalBar}{Averagerightleft}{\def\verticalbarAveragerightleft{#1}}

%%%%%%%%%%%%%%%%%%%%%%%%%%%%%%%%%%%%%%%%%%%% Default command %%%%%%%%%%%%%%%%%%%%%%%%%%%%%%%%%%%%%%%%%%%
\newcommand{\suffix}{\relax} 
\newcommand{\prefix}{\relax}
\newcommand{\VerticalBarXlabelrotate}{0}%default
\newcommand{\VerticalBartextwidth}{2.2}%default
\newcommand{\VerticalBarAlign}{center}%default
\newcommand{\VerticalBarNodeFontSize}{\small}%default
\newcommand{\VerticalBarFontSize}{\normalsize}
\newcommand{\VerticalBarAnchor}{center}%default
\newcommand{\VerticalBarYshift}{0}%default
\newcommand{\VerticalBarXshift}{0}%default
\newcommand{\VerticalBarLegendYshift}{0}%default
\newcommand{\VerticalBarLegendXshift}{0}%default
\newcommand{\VerticalBarXlimitEnLarge}{0.5}
\newcommand{\choice}{Print}
\newcommand{\verticalbartitleupdown}{1}
\newcommand{\verticalbarAveragerightleft}{-4}
%%%%%%%%%%%%%%%%%%%%%%%%%%%%%%%%%%%%%%%%%%%%%%%%%%%%%%%%%%%%%%%%%%%%%%%%%%%%%%%%%%%%%%%%%%%%%%%%%%%%%%%%%%%%
\newcommand{\VerticalBar}[4]{
 \setkeys{VerticalBar}{#1}

\def \barcoords{\VerticalBarcoords}    

    \pgfmathsetmacro{\numOfBars}{{\barcoords}[0]}     
    \pgfmathsetmacro{\noOfCoords}{{\barcoords}[1]}     
    \centering
    \begin{tikzpicture}
    [scale=\VerticalBarscale]
                \begin{axis}[
                    width=\VerticalBarwidth cm,
                    height=\VerticalBarheight cm,
                    bar width=\VerticalBarbarwidth cm,
                    tick style={draw=none},
                    axis x line*=bottom,
                    axis y line*=left,
                    % Modified x-axis setup to show all labels
                    xtick=\VerticalBarXtick,
                    xtick distance=1,
                    x tick label as interval=false,
                    xticklabels = {#4},
                    % xticklabels/.expanded=\VerticalBarXticklabels,
                    xticklabel style={
                    rotate =\VerticalBarXlabelrotate , 
                    anchor=\VerticalBarAnchor,
                    text width=\VerticalBartextwidth cm,
                    align=\VerticalBarAlign,
                    font=\VerticalBarNodeFontSize,
                    yshift=\VerticalBarYshift cm,
                    xshift=\VerticalBarXshift cm},
                    % ,rotate=10, anchor=east
                    ytick=\VerticalBarYtick,
                    yticklabel style={
                        font = \VerticalBarFontSize
                    },
                    ybar,
                    ymin=\VerticalBarYMin,
                    ymax=\VerticalBarYMax,
                    area legend,
                    ylabel={\VerticalBarFontSize \VerticalBarYlabels},
                    xlabel={\VerticalBarFontSize \VerticalBarXlabel},
                    nodes near coords={
                         \pgfmathprintnumber[precision=0]{\pgfplotspointmeta}\suffix
                     },
                    nodes near coords style={font=\VerticalBarNodeFontSize},
                    nodes near coords align={vertical},
                    legend style={at={(\VerticalBarLegendXposition,\VerticalBarLegendYposition)}, draw=none, anchor=north,legend columns=-1,column sep=0.2cm , font = \VerticalBarFontSize},
                    legend image code/.code={
                    \draw[] (0cm,-0.3cm) rectangle (1cm,0.3cm);
                    },
                    ymajorgrids=true,
                    xmajorgrids=false,
                    extra x ticks={1.5,2.5,3.5,4.5,5.5,6.5,7.5,8.5,9.5,10.5,11.5,12.5,13.5,14.5,15.5,16.5,17.5,18.5}, % Midpoints between x=1,2,3,4,5
                    extra x tick labels={},
                    extra x tick style={
                        grid=major, 
                        grid style={dashed, gray!50} % Customize as needed
                    },
                    title=\textbf{\VerticalBarFontSize\VerticalBartitle},
                    title style={
                        yshift=\verticalbartitleupdown cm, 
                        align=center 
                    },
                    % Ensure all x positions are shown
                    enlarge x limits={abs=\VerticalBarXlimitEnLarge}                    
                ]

        \colorlet{Color1}{blue!30}
        \definecolor{Color2}{RGB}{70,189,198}%
        \definecolor{Color3}{RGB}{248,201,40}%



        \foreach \i in {1,...,\numOfBars}{
            \coords{}
        
            \foreach \x in {0,...,\inteval{\noOfCoords-1}}{
                \pgfmathsetmacro{\eachpoint}{{\barcoords}[\i+1][\x]}
                \global\coords\expandafter{\expanded{\the\coords(\x+1,\eachpoint)}}
            }
            \edef \plotting{ \noexpand
                \addplot[ 
                    draw=none, 
                    fill=Color\i
                ] coordinates {
                    \the\coords
                }; \noexpand
            }
            
            \plotting

            \renewcommand{\coords}{{}}
        }
         % average line
        \def\halfBarWidth{\VerticalBarbarwidth}
        \def\avg{#3} 
        \ifx\avg\empty{}
        \else
        {
            \draw[thick, dashed, color=red!80] (0,\avg) -- (\noOfCoords + \halfBarWidth + .5,\avg)node[above,xshift=\verticalbarAveragerightleft cm] {\VerticalBarFontSize\prefix \pgfmathprintnumber{\avg} \suffix};
        }
        \fi

        \legend{#2}
                \end{axis}
            \end{tikzpicture}
}


%%____________________________________________________________________________

\makeatother

%%%%%%%%%%%%%%%%%%%%%%%%%%%%%%%%%%%%%%%%%%%%%%%%%%%%%%%%%%%%%%%

%%%%%%%%%%%%%%%%%%%%%%%%%%%%%%%%%%%%%%%%%%%%%%%%%%%%%%%%%%%%%%%%%%%%%%
%                      Math Vertical Bar
%%%%%%%%%%%%%%%%%%%%%%%%%%%%%%%%%%%%%%%%%%%%%%%%%%%%%%%%%%%%%%%%%%%%%%

\newtoks \coords 
\makeatletter

%%%%%%%%%%%%%%%%%%%%%%%%%%%%%%%%%%%  dimensions  %%%%%%%%%%%%%%%%%%%%%%%%%%%%%%%%%%%%%%%%%%%
\define@key{MathVerticalBar}{Width}{\def\MathVerticalBarwidth{#1}} % Width of Chart
\define@key{MathVerticalBar}{Height}{\def\MathVerticalBarheight{#1}}% Height of Chart
\define@key{MathVerticalBar}{Barwidth}{\def\MathVerticalBarbarwidth{#1}}% BarWidth of Chart
\define@key{MathVerticalBar}{Textwidth}{\def\MathVerticalBartextwidth{#1}}% Text width from axisline
\define@key{MathVerticalBar}{Scale}{\def\MathVerticalBarscale{#1}}% Size of Chart 
\define@key{MathVerticalBar}{YMax}{\def\MathVerticalBarYMax{#1}} % Y Axis Max Value
\define@key{MathVerticalBar}{YMin}{\def\MathVerticalBarYMin{#1}} % Y Axis Min Value

%%%%%%%%%%%%%%%%%%%%%%%%%%%%%%%%%%%%%% labels  %%%%%%%%%%%%%%%%%%%%%%%%%%%%%%%%%%%%%%%%%%%%%
% \define@key{MathVerticalBar}{Xticklabels}{\def\MathVerticalBarXticklabels{#1}} % X axis Tick Labels 
\define@key{MathVerticalBar}{Ylabels}{\def\MathVerticalBarYlabels{#1}} % Y axis labels
\define@key{MathVerticalBar}{Xlabel}{\def\MathVerticalBarXlabel{#1}} % X axis label
\define@key{MathVerticalBar}{Xlabelrotate}{\def\MathVerticalBarXlabelrotate{#1}} % X tick label rotate
%%%%%%%%%%%%%%%%%%%%%%%%%%%%%%%%%%%%%% Axis Values  %%%%%%%%%%%%%%%%%%%%%%%%%%%%%%%%%%%%%%%%%%%%%
\define@key{MathVerticalBar}{Xtick}{\def\MathVerticalBarXtick{#1}}% X axis tick values
\define@key{MathVerticalBar}{Ytick}{\def\MathVerticalBarYtick{#1}}% Y axis tick values
\define@key{MathVerticalBar}{Barcoords}{\def\MathVerticalBarcoords{#1}}% Bar cordinates

%%%%%%%%%%%%%%%%%%%%%%%%%%%%%%%%%%%%%%%%%%%%%%%%%%%%%%%%%%%%%%%%%%%%%%%%%%%%%%%%%%%%%%%%%%%%%%%%
\define@key{MathVerticalBar}{Fontsize}{\def\MathVerticalBarFontSize{#1}}
\define@key{MathVerticalBar}{Title}{\def\MathVerticalBartitle{#1}} % Title of Chart
\define@key{MathVerticalBar}{Suffix}{\def\mathsuffix{#1}}
\define@key{MathVerticalBar}{Prefix}{\def\mathprefix{#1}}
\define@key{MathVerticalBar}{Align}{\def\MathVerticalBarAlign{#1}}
\define@key{MathVerticalBar}{Anchor}{\def\MathVerticalBarAnchor{#1}}
\define@key{MathVerticalBar}{Yshift}{\def\MathVerticalBarYshift{#1}}
\define@key{MathVerticalBar}{Xshift}{\def\MathVerticalBarXshift{#1}}
\define@key{MathVerticalBar}{LegendTopBottom}{\def\MathVerticalBarLegendYposition{#1}}
\define@key{MathVerticalBar}{LegendLeftRight}{\def\MathVerticalBarLegendXposition{#1}}

\newcommand{\mathsuffix}{\relax} 
\newcommand{\mathprefix}{\relax}
\newcommand{\MathVerticalBarXlabelrotate}{0}%default
\newcommand{\MathVerticalBartextwidth}{2.2}%default
\newcommand{\MathVerticalBarAlign}{center}%default
\newcommand{\MathVerticalBarFontSize}{\normalsize}%default
\newcommand{\MathVerticalBarAnchor}{center}%default
\newcommand{\MathVerticalBarYshift}{0}%default
\newcommand{\MathVerticalBarXshift}{0}%default
\newcommand{\MathVerticalBarLegendYshift}{0}%default
\newcommand{\MathVerticalBarLegendXshift}{0}%default

\newcommand{\MathVerticalBar}[4]{
 \setkeys{MathVerticalBar}{#1}

\def \barcoords{\MathVerticalBarcoords}    

    \pgfmathsetmacro{\numOfBars}{{\barcoords}[0]}     
    \pgfmathsetmacro{\noOfCoords}{{\barcoords}[1]}     
    \centering
    \begin{tikzpicture}
    [scale=\MathVerticalBarscale]
                \begin{axis}[
                    width=\MathVerticalBarwidth cm,
                    height=\MathVerticalBarheight cm,
                    bar width=\MathVerticalBarbarwidth cm,
                    tick style={draw=none},
                    axis x line*=bottom,
                    axis y line =none,
                    ymajorgrids = true,
                    % Modified x-axis setup to show all labels
                    xtick=\MathVerticalBarXtick,
                    xtick distance=1,
                    x tick label as interval=false,
                    xticklabels = {#4},
                    % xticklabels/.expanded=\MathVerticalBarXticklabels,
                    xticklabel style={
                    rotate =\MathVerticalBarXlabelrotate , 
                    anchor=\MathVerticalBarAnchor,
                    text width=\MathVerticalBartextwidth cm,
                    align=\MathVerticalBarAlign,
                    font=\MathVerticalBarFontSize,
                    yshift=\MathVerticalBarYshift cm,
                    xshift=\MathVerticalBarXshift cm},
                    % ,rotate=10, anchor=east
                    ytick=\MathVerticalBarYtick,
                    yticklabel = {\empty},
                    ybar= -.5pt,
                    ymin=\MathVerticalBarYMin,
                    ymax=\MathVerticalBarYMax,
                    area legend,
                    % ylabel={\MathVerticalBarYlabels},
                    % ylabel style={
                        % yshift= -0.65cm
                    % },
                    xlabel={\MathVerticalBarXlabel},
                    nodes near coords={
                         \pgfmathprintnumber[precision=0]{\pgfplotspointmeta}\mathsuffix
                     },
                    nodes near coords style={font=\footnotesize,anchor= center ,rotate = 90,xshift = 3mm},
                    nodes near coords align={vertical},
                    legend style={at={(\MathVerticalBarLegendXposition,\MathVerticalBarLegendYposition)}, draw=none, font= \Large ,anchor=north,legend columns=-1,column sep=0.2cm},
                    title=\textbf{\large\MathVerticalBartitle},
                    % Ensure all x positions are shown
                    enlarge x limits={abs=0.5}                    
                ]

        \colorlet{Color1}{blue!30}
        \colorlet{Color2}{green!50}

        \foreach \i in {1,...,\numOfBars}{
            \coords{}
        
            \foreach \x in {0,...,\inteval{\noOfCoords-1}}{
                \pgfmathsetmacro{\eachpoint}{{\barcoords}[\i+1][\x]}
                \global\coords\expandafter{\expanded{\the\coords(\x+1,\eachpoint)}}
            }
            \edef \plotting{ \noexpand
                \addplot[ 
                    draw=none, 
                    fill=Color\i
                ] coordinates {
                    \the\coords
                }; \noexpand
            }
            
            \plotting

            \renewcommand{\coords}{{}}
        }
         % average line
        \def\halfBarWidth{\MathVerticalBarbarwidth}
        \def\avg{#3} 
        \ifx\avg\empty{}
        \else
        {
            \draw[thick, dashed, color=red!80] (0,\avg) -- (\noOfCoords + \halfBarWidth,\avg)node[above,xshift=-4 cm] {\mathprefix \pgfmathprintnumber{\avg} \mathsuffix};
        }
        \fi

        \legend{#2}
                \end{axis}
            \end{tikzpicture}
}

\makeatother




%%%%%%%%%%%%%%%%%%%%%%%%%%%%%%%%%%%%%%%%%%%%%%%%%%%%%%%%%%%%%%%
%       Question Response Circlula Bar Plot
%%%%%%%%%%%%%%%%%%%%%%%%%%%%%%%%%%%%%%%%%%%%%%%%%%%%%%%%%%%%%%%

% #1 - width (degree) of each sector 360/60(number of question)
% #2 - min distance from center in mm
% #3 - max distance from center in cm
% #4 - questio data in Que no/{wrong,learn_later,correct}
% #5 - Class 
% %6 - Subject

\newcommand{\drawLegends}[2]{
            \centering
            \begin{tikzpicture}[baseline=(label.base), scale=0.9] % Scale can also be used to reduce size
                \node[rectangle, draw, inner sep=0.5pt, fill=#2] (label) {\tiny \textcolor{#2}{11}};
            \end{tikzpicture}
            {\small #1} % Adjust text size here if needed
            \hspace{0.2cm} % Reduce spacing
        }

\newcommand{\queresponselegend}{\centering
        \scriptsize
        \drawLegends{Correct}{green!50}
        \drawLegends{Wrong}{red!30}
        \drawLegends{Learn Later}{yellow!30!white}\\
        \vspace{0.5cm}
        \parbox[c]{7cm}{
        \begin{scriptsize}
        \RaggedRight
        \highyellowred{Q} - Performance <20\% \\
        \vspace{0.2cm}
        \highyellow{Q} - Performance between 20\% and 40\% \\
        \vspace{0.2cm}
        \highgreen{Q} - Performance > 85\% \\
        \end{scriptsize}}}
        
\newcommand{\queresponsecirculatchart}[6]{
\centering
\tikzset{
        hist 1/.style={fill=red!30}, %wrong
        hist 2/.style={fill=yellow!30!white}, %learn later
        hist 3/.style={fill=green!50}, %correct
      }

      \def\astep{#1} % width (degree) of each sector 360/60(number of question)
      \def\minA{#2mm} % min distance from center
      \def\maxA{#3cm} % max distance from center

      \begin{tikzpicture}[scale = 0.8,text=black,font=\footnotesize,]
        \fill[white] circle(\maxA+1cm);

        \foreach \curlabel/\values [count=\cp] in {#4}
          % ... (omitting other data for brevity)
          {
          \ifthenelse{\equal{\curlabel}{}}{}{
            % angle for this current label
            \pgfmathsetmacro{\angle}{(\cp-1)*\astep-90+\astep/2}
            % distance from center
            \pgfmathsetmacro{\total}{\minA}
            \pgfmathsetmacro{\am}{\maxA-\minA}
            \xdef\total{\total}
            % histogram
            \foreach \val [count=\cv] in \values {
              \pgfmathsetmacro{\nexttotal}{\total pt+\am/100*\val}
              % sector
              \filldraw[hist \cv]
              (\angle+\astep/2:\total pt)
              arc(\angle+\astep/2:\angle-\astep/2:\total pt)
              -- (\angle-\astep/2:\nexttotal pt)
              arc(\angle-\astep/2:\angle+\astep/2:\nexttotal pt)
              -- cycle;
              % iteration
              \xdef\total{\nexttotal}
              \typeout{\val:\total}
            }
            % label (with autorotation)
            \pgfmathtruncatemacro{\anglenode}{\angle}
            \ifthenelse{\( \anglenode > 90 \) \AND \( \anglenode < 270 \)}{ 
              \node[rotate=180+\anglenode,anchor=east,font=\tiny] at (\angle:\maxA) {\curlabel};
            }{
              \node[rotate=\anglenode,anchor=west,font=\tiny] at (\angle:\maxA) {\curlabel};
            }
          }
        }
        \node[align=center, font=\bfseries\scriptsize] at (0,0) {#5 \\ #6};
      \end{tikzpicture}}



%%%%%%%%%%%% To Optimised
%%%%%%%%%%%%%%%%%%%%%%%%%%%%%%%%%%%%%%%%%%%%%%%%%%%%%%%%%%%%%%%%%%%%%%%%%%%%%%%%

%                   Student Percentage Distribution Analysis

%%%%%%%%%%%%%%%%%%%%%%%%%%%%%%%%%%%%%%%%%%%%%%%%%%%%%%%%%%%%%%%%%%%%%%%%%%%%%%%%

% #1 - Height of the graph in (cm)
% #2 - Width of the graph in (cm)
% #3 - Bar width
% #4 - ymax value
% #5 - X axis tick labels with separations
% #6 - Y-axis Label
% #7 - Graph legends
% #8 - No of Graph lines, No of plots and Coordinates of the graph
% Array 0 - No of legends
% Array 1 - No of plots
% Array 2 - Coordinates of the graph upto 
% Array n - Coordinates of the graph
% #9 - Title of the graph

\newcommand{\studentDistribution}[9]{
    \def \barcoords{#8}    % No of bars, No of plots and Coordinates of the graph

    \pgfmathsetmacro{\numOfBars}{{\barcoords}[0]}     % No of bars
    \pgfmathsetmacro{\noOfCoords}{{\barcoords}[1]}     % No of plots
    
    \begin{tikzpicture}[scale=0.4]
    \centering
    \begin{axis}[
        ybar=0pt, axis on top,
        title={\Large #9},
        height=#1 cm, 
        width=#2 cm,
        bar width=#3 cm,
        ymajorgrids, tick align=outside,
        major grid style={draw=black,opacity=0.8,dotted},
        enlarge y limits={value=.1,upper},
        ymin=0, 
        ymax=#4, % Adjusted to fit the data range
        axis x line*=bottom,
        axis y line*=left,
        y axis line style={opacity=1},
        tickwidth=0pt,
        enlarge x limits=0.1,
        legend style={
            at={(0.5,-0.2)},
            anchor=north,
            legend columns=-1,
            /tikz/every even column/.append style={column sep=0.5cm},
            draw=none
        },
        ylabel={\Large #6},
        xticklabels={#5},
       xtick=data,
       xticklabel style={anchor=north,xshift=6mm, font=\small},
       nodes near coords={
        \pgfmathprintnumber[precision=0]{\pgfplotspointmeta}
       },
       legend image code/.code={
            \draw [] (0cm,-0.15cm) rectangle (0.15cm,0.15cm);
        },
       xticklabel style={font=\small} % Reducing the text size of X-axis labels
    ]

        \foreach \i in {1,...,\numOfBars}{     % Loop for stacks
            \coords{}
        
            \foreach \x in {0,...,\inteval{\noOfCoords-1}}{     % Loop for plots
                \pgfmathsetmacro{\eachpoint}{{\barcoords}[\i+1][\x]}
                \global\coords\expandafter{\expanded{\the\coords(\x+1,\eachpoint)}}
                % Declare the \coords as global and make it as tokens
            }

            \renewcommand{\coords}{{}}  % Clear the \coords data for storing the other stack coords
        }
        \addplot [draw=blue!60, fill=blue!30] coordinates {\the\coords};
      
        \addplot [smooth, update limits=false, red, thick, mark=*] coordinates {\the\coords};

        \legend{#7}
  \end{axis}
\end{tikzpicture}
}

%%%%%%%%%%%%%%%%%%%%%%%%%%%%%%%%%%%%%%%%%%%%%%%%%%%%%%%%%%%%%%%%%%%%%%%%%%%%%%%

%                   Student Percentage Distribution Analysis

%%%%%%%%%%%%%%%%%%%%%%%%%%%%%%%%%%%%%%%%%%%%%%%%%%%%%%%%%%%%%%%%%%%%%%%%%%%%%%%%

% #1 - Height of the graph in (cm)
% #2 - Width of the graph in (cm)
% #3 - Bar width
% #4 - ymax value
% #5 - X axis tick labels with separations
% #6 - Y-axis Label
% #7 - Graph legends
% #8 - No of Graph lines, No of plots and Coordinates of the graph
% Array 0 - No of legends
% Array 1 - No of plots
% Array 2 - Coordinates of the graph upto 
% Array n - Coordinates of the graph
% #9 - Title of the graph

\newcommand{\studentDistributionwithoutline}[9]{
    \def \barcoords{#8}    % No of bars, No of plots and Coordinates of the graph

    \pgfmathsetmacro{\numOfBars}{{\barcoords}[0]}     % No of bars
    \pgfmathsetmacro{\noOfCoords}{{\barcoords}[1]}     % No of plots
    
    \begin{tikzpicture}[scale=0.4]
    \centering
    \begin{axis}[
        ybar=0pt, axis on top,
        title={\Large \Large #9},
        height=#1 cm, 
        width=#2 cm,
        bar width=#3 cm,
        ymajorgrids, tick align=outside,
        major grid style={draw=black,opacity=0.8,dotted},
        enlarge y limits={value=.1,upper},
        ymin=0, 
        ymax=#4, % Adjusted to fit the data range
        axis x line*=bottom,
        axis y line*=left,
        y axis line style={opacity=1},
        tickwidth=0pt,
        enlarge x limits=0.1,
        legend style={
            at={(0.5,-0.2)},
            anchor=north,
            legend columns=-1,
            /tikz/every even column/.append style={column sep=0.5cm},
            draw=none
        },
        ylabel={\Large #6},
        xticklabels={#5},
        xticklabel style={anchor=north,xshift=6mm, font=\small},
       xtick=data,
       nodes near coords={
        \pgfmathprintnumber[precision=0]{\pgfplotspointmeta}
       },
       legend image code/.code={
            \draw [] (0cm,-0.15cm) rectangle (0.15cm,0.15cm);
        },
       xticklabel style={font=\small} % Reducing the text size of X-axis labels
    ]

        \foreach \i in {1,...,\numOfBars}{     % Loop for stacks
            \coords{}
        
            \foreach \x in {0,...,\inteval{\noOfCoords-1}}{     % Loop for plots
                \pgfmathsetmacro{\eachpoint}{{\barcoords}[\i+1][\x]}
                \global\coords\expandafter{\expanded{\the\coords(\x+1,\eachpoint)}}
                % Declare the \coords as global and make it as tokens
            }

            \renewcommand{\coords}{{}}  % Clear the \coords data for storing the other stack coords
        }
        \addplot [draw=blue!60, fill=blue!30] coordinates {\the\coords};
      
        % \addplot [sharp plot, update limits=false, red, thick, mark=*] coordinates {\the\coords};

        \legend{#7}
  \end{axis}
\end{tikzpicture}
}

%%%%%%%%%%%%%%%%%%%%%%%%%%%%%%%%%%%%%%%%%%%%%%%%%%%%%%%%%%%%%%%%%%%%%%

%                       Impact of Fundamentals

%%%%%%%%%%%%%%%%%%%%%%%%%%%%%%%%%%%%%%%%%%%%%%%%%%%%%%%%%%%%%%%%%%%%%%
% #1 - Height of the graph in (cm)
% #2 - Width of the graph in (cm)
% #3 - Bar width
% #4 - X axis tick labels with separations
% #5 - Y-axis Label
% #6 - Graph legends
% #7 - No of Graph lines, No of plots and Coordinates of the graph
% Array 0 - No of legends
% Array 1 - No of plots
% Array 2 - Coordinates of the graph upto
% Array n - Coordinates of the graph
% #8 - Title of the graph
% #9 - total sum value
\newtoks \coords 

\newcommand{\impactOfFundamentals}[9]{
    \def \barcoords{#7}
    \pgfmathsetmacro{\numOfStacks}{{\barcoords}[0]}    
    \pgfmathsetmacro{\noOfCoords}{{\barcoords}[1]}

    \begin{tikzpicture}[scale=0.78]
        \begin{axis}[
            ybar stacked,
            axis on top,
            title={#8},
            height=#1 cm, 
            width=#2 cm,
            bar width=#3 cm,
            ymajorgrids, tick align=outside,
            major grid style={draw=black,opacity=0.8,dotted},
            enlarge y limits={value=.1,upper},
            ymin=0, ymax=100,
            axis x line*=bottom,
            axis y line*=right,
            y axis line style={opacity=0},
            tickwidth=0pt,
            enlarge x limits=.3,
            yticklabel={$\pgfmathprintnumber{\tick}\%$},
            legend style={
                at={(0.5,-0.1)},
                anchor=north, 
                legend columns=-1,
                /tikz/every even column/.append style={column sep=0.5cm},
                draw=none
            },
            ylabel={#5},
            xticklabels={#4},
            xticklabel style = {align=center},
            xtick=data,
            nodes near coords align={center},
            nodes near coords={\pgfmathprintnumber[precision=0]{\pgfplotspointmeta}\%},
            legend image code/.code={
                \draw [] (0cm,-0.15cm) rectangle (0.15cm,0.15cm);
            },
            legend style={at={(0.5,-0.2)}, draw=none, anchor=north,legend columns=-1,column sep=0.2cm , font = \VerticalBarFontSize},
            after end axis/.code={
                \pgfmathsetmacro{\i}{1}
                \foreach \val in #9 {
                    \node[anchor=south, font=\small\bfseries, yshift=3pt] 
                        at (axis cs:\i, \val) {\pgfmathprintnumber[precision=0]{\val}\%};
                    \pgfmathparse{\i + 1} \global\let\i=\pgfmathresult
                }
            }
        ]

        \foreach \i in {1,...,\numOfStacks}{
            \coords{}
            \foreach \x in {0,...,\inteval{\noOfCoords-1}}{
                \pgfmathsetmacro{\eachpoint}{{\barcoords}[\i+1][\x]}
                \global\coords\expandafter{\expanded{\the\coords(\x+1,\eachpoint)}}
            }
            \ifnum \i = 1
                \addplot [draw=blue!30, fill=blue!30] coordinates { \the\coords };
            \else
                \addplot [pattern=north east lines, draw=blue!30, pattern color=blue!40] coordinates { \the\coords };
            \fi
            \renewcommand{\coords}{{}} 
        }

        \legend{#6}
        \end{axis}
    \end{tikzpicture}
}



%%%%%%%%%%%%%%%%%%%%%%%%%%%%%%%%%%%%%%%%%%%%%%%%%%%%%%%%%%%%%%%%%%%%%%%%%%%%%%%%%%%%%%%%%%%%


%%%%%%%%%%%%%%%%%%%%%%%%%%%%%%%%%bidirectional graph


\newtoks \basecoords    % Tokenization for the base coordinate values


%   #1 -- Labels
%   #2 -- Bar Width in cm
%   #3 -- Scale value in cm
%   #4 -- distance of the node value from the coordinates in cm
%   #5 -- Bar Labels
%   #6 -- No.of Plots
%   #7 -- Coordinate values


\newcommand{\bidirectionalBarGraph}[9]{
    \def \labeltext{#1}
    \pgfmathsetmacro{\leftlabel}{{\labeltext}[0]}     
    \pgfmathsetmacro{\rightlabel}{{\labeltext}[1]}

    \def \labelbars{#5}
    \def \nodevaluedistance{#4}
    \def \numofplots{#6}
    \def \barcoords{#7}

    \pgfmathsetmacro{\mathaverage}{#8/10}
    \pgfmathsetmacro{\scienceaverage}{#9/10}
    
    % ----------------------------------------------
    % Defining custom colors 
    \colorlet{arrowColor}{gray!70}
    \colorlet{coordtextColor}{black!70}
    \colorlet{leftbarsColor}{orange!47}
    \colorlet{rightbarsColor}{blue!50}
    \colorlet{trendline}{red!45}
    % ----------------------------------------------
    
    \begin{tikzpicture}[
        labels/.style={
            text=black,
            font=\tiny,
            fill=none
        },
        basebarstyle1/.style={
            pattern=north east lines, 
            pattern color=leftbarsColor,
            draw=leftbarsColor,
            opacity=0.6
        },
        basebarstyle2/.style={
            pattern=north west lines, 
            pattern color=rightbarsColor,
            draw=rightbarsColor,
            opacity=0.6
        },
        barstyle1/.style={ 
            fill=leftbarsColor,
            draw=leftbarsColor
        },
        barstyle2/.style={
            fill=rightbarsColor,
            draw=rightbarsColor
        },
        plotting/.style={
            xbar, 
            bar width=#2 cm
        },
        barlabels/.style={
            text=black, 
            font=\tiny,
            fill=white!10,
            text opacity=10,
            align=center
        },
        coordlabels/.style={
            text=coordtextColor, 
            font=\tiny\bfseries, 
            fill=none, 
            text opacity=1,
            align=center
        },
        scale=#3
    ]
    
    % Defining the left and right labels
    \node at (-2,\numofplots+0.2)[labels]{\large \textcolor{orange}{\leftlabel}};
    \node at (2,\numofplots+0.2)[labels]{\large \textcolor{blue}{\rightlabel}};

    \foreach \i in {1,...,2}{
        \coords{}
        \basecoords{}
        
        % Reverse loop for plotting bars top-to-bottom
        \foreach \j in {0,...,\inteval{\numofplots-1}}{
            \pgfmathsetmacro{\eachpoint}{{\barcoords}[\i-1][\j]}
            \pgfmathsetmacro{\reducedpoint}{\eachpoint*0.1}
            
            \pgfmathsetmacro{\reverseIndex}{\numofplots-\j-1} % Calculate reverse index
            
            \ifnum \i = 1
                \global\coords\expandafter{\expanded{\the\coords(-\reducedpoint,\reverseIndex)}}
                \global\basecoords\expandafter{\expanded{\the\basecoords(-10,\reverseIndex)}}
            \else
                \global\coords\expandafter{\expanded{\the\coords(\reducedpoint,\reverseIndex)}}
                \global\basecoords\expandafter{\expanded{\the\basecoords(10,\reverseIndex)}}
            \fi
        }

        \edef \visual{ \noexpand
            \draw[behind path, barstyle\i] plot[plotting] coordinates {
                \the\coords
            }; \noexpand  
            \draw[basebarstyle\i] plot[plotting] coordinates {
                \the\basecoords
            }; \noexpand
        }
        
        \visual
        \renewcommand{\basecoords}{{}}
        \renewcommand{\coords}{{}}
    }

    % Drawing center axis
    \draw[<->, color=arrowColor](0,-1) -- (0,\numofplots);



    % Trend Lines
    \pgfmathsetmacro{\sciencepercentage}{\scienceaverage*10}
    \draw[thick, dashed, color=trendline](\scienceaverage,-0.8) -- (\scienceaverage,\numofplots-0.2)node[below,pos= 0,xshift=2mm] { \footnotesize \textcolor{red}{Average = \pgfmathprintnumber{\sciencepercentage}\%}}; % right
    \pgfmathsetmacro{\mathpercentage}{\mathaverage*10}
    \draw[thick, dashed, color=trendline](-\mathaverage,-0.8) -- (-\mathaverage,\numofplots-0.2)node[below,pos= 0,xshift=-2mm] { \footnotesize \textcolor{red}{Average = \pgfmathprintnumber{\mathpercentage}\%}}; % left

    % Adding coordinate labels
    \foreach \x in {1,...,2}{
        \foreach \y in {0,...,\inteval{\numofplots-1}}{
            \pgfmathsetmacro{\eachpoint}{{\barcoords}[\x-1][\y]}
            \pgfmathsetmacro{\reducedpoint}{\eachpoint*0.1}
            \pgfmathsetmacro{\reverseIndex}{\numofplots-\y-1} % Reverse index
            
            \ifnum \x = 1
                \ifnum \eachpoint > 0
                    \ifnum \eachpoint < 95
                        \node at (-\reducedpoint-\nodevaluedistance-.2,\reverseIndex)[coordlabels]{\eachpoint\%};
                    \else
                        \node at (-\reducedpoint+\nodevaluedistance,\reverseIndex)[coordlabels]{\eachpoint\%};
                    \fi
                \fi
            \else
                \ifnum \eachpoint > 0
                    \ifnum \eachpoint < 95
                        \node at (\reducedpoint+\nodevaluedistance+.2,\reverseIndex)[coordlabels]{\eachpoint\%};
                    \else
                        \node at (\reducedpoint-\nodevaluedistance,\reverseIndex)[coordlabels]{\eachpoint\%};
                    \fi
                \fi
            \fi
        }
    }

    % Adding bar labels
    \foreach \x in {0,...,\inteval{\numofplots-1}}{
        \pgfmathsetmacro{\reverseIndex}{\numofplots-\x-1} % Reverse index
        \pgfmathsetmacro \currentlabel{{\labelbars}[\x]}
        \node at (0,\reverseIndex)[barlabels]{{\currentlabel}};
    }

    \end{tikzpicture}
}



%%%%%%%%%%%%%%%%%%%%%%%%%%%%%%%% unidirectional Graph %%%%%%%%%%%%%%%%%%%%%%%%

\newcommand{\unidirectionalBarGraph}[8]{
    \def \labeltext{#1}
    \def\leftlabel{\labeltext}  
    \def \labelbars{#5}
    \def \nodevaluedistance{#4}
    \def \numofplots{#6}
    \def \barccoords{#7}
    \pgfmathsetmacro{\mathaverage}{#8/10}
    
    \colorlet{arrowColor}{gray!70}
    \colorlet{coordtextColor}{black!70}
    \colorlet{leftbarsColor}{blue!47}
    \colorlet{trendline}{red!45}
    
    \begin{tikzpicture}[
        labels/.style={
            text=black,
            font=\tiny,
            fill=none
        },
        basebarstyle1/.style={
            pattern=north east lines, 
            pattern color=leftbarsColor,
            draw=leftbarsColor,
            opacity=0.6
        },
        barstyle1/.style={ 
            fill=leftbarsColor,
            draw=leftbarsColor
        },
        plotting/.style={
            xbar, 
            bar width=#2 cm
        },
        barlabels/.style={
            text=black, 
            font=\tiny,
            fill=white!10,
            text opacity=10,
            align=center
        },
        coordlabels/.style={
            text=coordtextColor, 
            font=\tiny\bfseries, 
            fill=none, 
            text opacity=1,
            align=center
        },
        scale=#3
    ]
    
    \node at (2,\numofplots+0.2)[labels]{\large \textcolor{blue}{\leftlabel}};

    \foreach \i in {1}{
        \def\ccoords{}
        \def\baseccoords{}
        
        \foreach \j in {0,...,\numexpr\numofplots-1\relax}{
            \pgfmathsetmacro{\eachpoint}{{\barccoords}[\j]}
            \pgfmathsetmacro{\reducedpoint}{\eachpoint*0.1}
            \pgfmathsetmacro{\reverseIndex}{\numofplots-\j-1}
            
            \xdef\ccoords{\ccoords (\reducedpoint,\reverseIndex)}
            \xdef\baseccoords{\baseccoords (10,\reverseIndex)}
        }

        \edef \visual{ 
            \noexpand\draw[behind path, barstyle1] plot[plotting] coordinates {\ccoords};  
            \noexpand\draw[basebarstyle1] plot[plotting] coordinates {\baseccoords};
        }
        \visual
    }

    \draw[-, color=arrowColor](0,-1) -- (0,\numofplots);

    \pgfmathsetmacro{\mathpercentage}{\mathaverage*10}
    \draw[thick, dashed, color=trendline](\mathaverage,-0.8) -- (\mathaverage,\numofplots-0.2)
        node[below,pos=0,xshift=-2mm] {\footnotesize \textcolor{red}{ Average = \pgfmathprintnumber{\mathpercentage}\%}};

    \foreach \y in {0,...,\numexpr\numofplots-1\relax}{
        \pgfmathsetmacro{\eachpoint}{{\barccoords}[\y]}
        \pgfmathsetmacro{\reducedpoint}{\eachpoint*0.1}
        \pgfmathsetmacro{\reverseIndex}{\numofplots-\y-1}
        \ifnum \eachpoint > 0
            \ifnum \eachpoint < 95
                \node at (\reducedpoint+0.6,\reverseIndex)[coordlabels]{\eachpoint\%};
            \else
                \node at (\reducedpoint+0.4,\reverseIndex)[coordlabels]{\eachpoint\%};
            \fi
        \fi
    }

    % Corrected student names loop
    \foreach \name [count=\x from 0] in \labelbars {
        \pgfmathsetmacro{\reverseIndex}{\numofplots-\x-1}
        \node at (0,\reverseIndex)[barlabels, anchor=east]{\name};
    }

    \end{tikzpicture}
}